\begin{frame}{Traditional k-Nearest Neighbor}
\setbeamercovered{invisible}
\begin{minipage}{.5\textwidth}
\small k-Nearest Neighbor:
\begin{itemize}
    \item<1-> \small Given:
    \begin{itemize}
        \item<2-> \small $q$: query point
        \item<3-> \small $T$: target set\\\tiny(e.g. $\{A,B,C,D\}$)
        \item<4-> \small $k$: number of retrieved targets \tiny(e.g. $k=1$)
    \end{itemize}
    \item<5-> \small Return:\\top $k$ targets regarding \textit{Euclidean distance} $d_e$
    \item<6-> \small the circle indicates that $D$ is the nearest neighbor of $q$
\end{itemize}
\end{minipage}%
\begin{minipage}{.5\textwidth}
\begin{adjustbox}{max totalsize={.9\textwidth}{.9\textheight}, right}
\begin{tikzpicture}[line cap=round,line join=round,>=triangle 45,x=1.0cm,y=1.0cm]
\definecolor{aqaqaq}{rgb}{0.6274509803921569,0.6274509803921569,0.6274509803921569}
\definecolor{qqqqff}{rgb}{0.,0.,1.}
\definecolor{ffqqqq}{rgb}{1.,0.,0.}
\clip(3.5,-2.7) rectangle (10.5,2.5);
\draw [line width=1.2pt,dash pattern=on 1pt off 1pt,color=aqaqaq,fill=aqaqaq,fill opacity=0.30000001192092896] (6.965791019668495,-0.8590646710374431) circle (1.226394681592311cm);
\begin{scriptsize}
\draw [fill=ffqqqq] (6.965791019668495,-0.8590646710374431) circle (2.5pt);
\draw[color=ffqqqq] (7.1713402737586955,-0.4852111547881992) node {$q$};
\draw [fill=qqqqff] (3.786645549990551,1.7424425037681481) circle (2.5pt);
\draw[color=qqqqff] (3.8990767613748694,2.0486954624936313) node {$A$};
\draw [fill=qqqqff] (7.240860537026145,0.8341118960661955) circle (2.5pt);
\draw[color=qqqqff] (7.355929497534194,1.1425301821411886) node {$B$};
\draw [fill=qqqqff] (6.786032179750889,-2.084374701009021) circle (2.5pt);
\draw[color=qqqqff] (6.5840109253821115,-2.4150075851684014) node {$D$};
\draw [fill=qqqqff] (8.744888255475496,0.3259148131345835) circle (2.5pt);
\draw[color=qqqqff] (8.866204964788267,0.6391050263898316) node {$C$};
\end{scriptsize}
\end{tikzpicture}
\end{adjustbox}
\end{minipage}
\end{frame}

\begin{frame}{Obstacle k-Nearest Neighbor}
\begin{minipage}{.5\textwidth}
\begin{itemize}
    \item \small traditional kNN has been well studied.
    \item \small when take obstacles into consideration...
    \item \small metric: Obstacle distance $d_o$
\end{itemize}
\end{minipage}%
\begin{minipage}{.5\textwidth}
\begin{adjustbox}{max totalsize={.9\textwidth}{.9\textheight}, right}
\only<1,2> {\begin{tikzpicture}[line cap=round,line join=round,>=triangle 45,x=1.0cm,y=1.0cm]
\definecolor{aqaqaq}{rgb}{0.6274509803921569,0.6274509803921569,0.6274509803921569}
\definecolor{qqqqff}{rgb}{0.,0.,1.}
\definecolor{ffqqqq}{rgb}{1.,0.,0.}
\clip(3.5,-2.7) rectangle (10.5,2.5);
\draw [line width=1.2pt,dash pattern=on 1pt off 1pt,color=aqaqaq,fill=aqaqaq,fill opacity=0.30000001192092896] (6.965791019668495,-0.8590646710374431) circle (1.226394681592311cm);
\begin{scriptsize}
\draw [fill=ffqqqq] (6.965791019668495,-0.8590646710374431) circle (2.5pt);
\draw[color=ffqqqq] (7.1713402737586955,-0.4852111547881992) node {$q$};
\draw [fill=qqqqff] (3.786645549990551,1.7424425037681481) circle (2.5pt);
\draw[color=qqqqff] (3.8990767613748694,2.0486954624936313) node {$A$};
\draw [fill=qqqqff] (7.240860537026145,0.8341118960661955) circle (2.5pt);
\draw[color=qqqqff] (7.355929497534194,1.1425301821411886) node {$B$};
\draw [fill=qqqqff] (6.786032179750889,-2.084374701009021) circle (2.5pt);
\draw[color=qqqqff] (6.5840109253821115,-2.4150075851684014) node {$D$};
\draw [fill=qqqqff] (8.744888255475496,0.3259148131345835) circle (2.5pt);
\draw[color=qqqqff] (8.866204964788267,0.6391050263898316) node {$C$};
\end{scriptsize}
\end{tikzpicture}}
\only<3-> {\begin{tikzpicture}[line cap=round,line join=round,>=triangle 45,x=1.0cm,y=1.0cm]
\definecolor{aqaqaq}{rgb}{0.6274509803921569,0.6274509803921569,0.6274509803921569}
\definecolor{qqqqff}{rgb}{0.,0.,1.}
\definecolor{ffqqqq}{rgb}{1.,0.,0.}

\clip(3.5,-2.7) rectangle (10.5,2.5);
\fill[line width=1.2pt,fill=black,fill opacity=0.8999999761581421] (4.273473007276684,-0.8915497346423199) -- (4.311125781159042,-1.6311118832188969) -- (10.071125781159033,-1.891111883218897) -- (10.011125781159034,-1.231111883218897) -- cycle;
\fill[line width=1.2pt,fill=black,fill opacity=1.0] (6.370910095846811,0.5910375080896166) -- (6.434877040051173,0.002541621409478362) -- (8.639743589985672,-0.6535907548495615) -- (8.748484255211462,0.3402661877139281) -- cycle;
\draw [line width=1.2pt,dash pattern=on 1pt off 1pt,color=aqaqaq,fill=aqaqaq,fill opacity=0.30000001192092896] (6.965791019668495,-0.8590646710374431) circle (1.226394681592311cm);
\draw [line width=1.2pt] (8.63151379877438,-0.6620475590162423)-- (8.744888255475496,0.3259148131345835);
\draw [line width=1.2pt] (6.965791019668495,-0.8590646710374431)-- (6.436308033635486,-0.035167994675882754);
\draw [line width=1.2pt] (6.965791019668495,-0.8590646710374431)-- (3.786645549990551,1.7424425037681481);
\draw [line width=1.2pt] (6.965791019668495,-0.8590646710374431)-- (4.274665113870528,-0.9032956187025807);
\draw [line width=1.2pt] (4.274665113870528,-0.9032956187025807)-- (4.306017252514734,-1.6439597133614232);
\draw [line width=1.2pt] (4.306017252514734,-1.6439597133614232)-- (6.786032179750889,-2.084374701009021);
\draw [line width=1.2pt] (6.434877040051173,0.002541621409478362)-- (6.370910095846811,0.5910375080896166);
\draw [line width=1.2pt] (6.370910095846811,0.5910375080896166)-- (7.240860537026145,0.8341118960661955);
\draw [line width=1.2pt] (6.965791019668495,-0.8590646710374431)-- (8.639743589985672,-0.6535907548495615);
\begin{scriptsize}
\draw [fill=ffqqqq] (6.965791019668495,-0.8590646710374431) circle (2.5pt);
\draw[color=ffqqqq] (7.1713402737586955,-0.4852111547881992) node {$q$};
\draw [fill=qqqqff] (3.786645549990551,1.7424425037681481) circle (2.5pt);
\draw[color=qqqqff] (3.8990767613748694,2.0486954624936313) node {$A$};
\draw [fill=qqqqff] (7.240860537026145,0.8341118960661955) circle (2.5pt);
\draw[color=qqqqff] (7.355929497534194,1.1425301821411886) node {$B$};
\draw [fill=qqqqff] (6.786032179750889,-2.084374701009021) circle (2.5pt);
\draw[color=qqqqff] (6.5840109253821115,-2.4150075851684014) node {$D$};
\draw [fill=qqqqff] (4.273473007276684,-0.8915497346423199) circle (2.5pt);
\draw [fill=qqqqff] (4.311125781159042,-1.6311118832188969) circle (2.5pt);
\draw [fill=qqqqff] (10.071125781159033,-1.891111883218897) circle (2.5pt);
\draw [fill=qqqqff] (10.011125781159034,-1.231111883218897) circle (2.5pt);
\draw [fill=qqqqff] (6.370910095846811,0.5910375080896166) circle (2.5pt);
\draw [fill=qqqqff] (6.434877040051173,0.002541621409478362) circle (2.5pt);
\draw [fill=qqqqff] (8.639743589985672,-0.6535907548495615) circle (2.5pt);
\draw [fill=qqqqff] (8.744888255475496,0.3259148131345835) circle (2.5pt);
\draw[color=qqqqff] (8.866204964788267,0.6391050263898316) node {$C$};
\end{scriptsize}
\end{tikzpicture}}
\end{adjustbox}
\end{minipage}
\end{frame}

\begin{frame}{Application Scenario}
\begin{minipage}{.5\textwidth}
\small In an industrial warehouse,\\
{\color{red}$q$} is a robot.\\
It's interested in the closest storage locations,\\
but it can not cross obstacles
\end{minipage}%
\begin{minipage}{.5\textwidth}
\begin{adjustbox}{max totalsize={.9\textwidth}{.9\textheight}, right}
\begin{tikzpicture}[line cap=round,line join=round,>=triangle 45,x=1.0cm,y=1.0cm]
\definecolor{aqaqaq}{rgb}{0.6274509803921569,0.6274509803921569,0.6274509803921569}
\definecolor{qqqqff}{rgb}{0.,0.,1.}
\definecolor{ffqqqq}{rgb}{1.,0.,0.}

\clip(3.5,-2.7) rectangle (10.5,2.5);
\fill[line width=1.2pt,fill=black,fill opacity=0.8999999761581421] (4.273473007276684,-0.8915497346423199) -- (4.311125781159042,-1.6311118832188969) -- (10.071125781159033,-1.891111883218897) -- (10.011125781159034,-1.231111883218897) -- cycle;
\fill[line width=1.2pt,fill=black,fill opacity=1.0] (6.370910095846811,0.5910375080896166) -- (6.434877040051173,0.002541621409478362) -- (8.639743589985672,-0.6535907548495615) -- (8.748484255211462,0.3402661877139281) -- cycle;
\draw [line width=1.2pt,dash pattern=on 1pt off 1pt,color=aqaqaq,fill=aqaqaq,fill opacity=0.30000001192092896] (6.965791019668495,-0.8590646710374431) circle (1.226394681592311cm);
\draw [line width=1.2pt] (8.63151379877438,-0.6620475590162423)-- (8.744888255475496,0.3259148131345835);
\draw [line width=1.2pt] (6.965791019668495,-0.8590646710374431)-- (6.436308033635486,-0.035167994675882754);
\draw [line width=1.2pt] (6.965791019668495,-0.8590646710374431)-- (3.786645549990551,1.7424425037681481);
\draw [line width=1.2pt] (6.965791019668495,-0.8590646710374431)-- (4.274665113870528,-0.9032956187025807);
\draw [line width=1.2pt] (4.274665113870528,-0.9032956187025807)-- (4.306017252514734,-1.6439597133614232);
\draw [line width=1.2pt] (4.306017252514734,-1.6439597133614232)-- (6.786032179750889,-2.084374701009021);
\draw [line width=1.2pt] (6.434877040051173,0.002541621409478362)-- (6.370910095846811,0.5910375080896166);
\draw [line width=1.2pt] (6.370910095846811,0.5910375080896166)-- (7.240860537026145,0.8341118960661955);
\draw [line width=1.2pt] (6.965791019668495,-0.8590646710374431)-- (8.639743589985672,-0.6535907548495615);
\begin{scriptsize}
\draw [fill=ffqqqq] (6.965791019668495,-0.8590646710374431) circle (2.5pt);
\draw[color=ffqqqq] (7.1713402737586955,-0.4852111547881992) node {$q$};
\draw [fill=qqqqff] (3.786645549990551,1.7424425037681481) circle (2.5pt);
\draw[color=qqqqff] (3.8990767613748694,2.0486954624936313) node {$A$};
\draw [fill=qqqqff] (7.240860537026145,0.8341118960661955) circle (2.5pt);
\draw[color=qqqqff] (7.355929497534194,1.1425301821411886) node {$B$};
\draw [fill=qqqqff] (6.786032179750889,-2.084374701009021) circle (2.5pt);
\draw[color=qqqqff] (6.5840109253821115,-2.4150075851684014) node {$D$};
\draw [fill=qqqqff] (4.273473007276684,-0.8915497346423199) circle (2.5pt);
\draw [fill=qqqqff] (4.311125781159042,-1.6311118832188969) circle (2.5pt);
\draw [fill=qqqqff] (10.071125781159033,-1.891111883218897) circle (2.5pt);
\draw [fill=qqqqff] (10.011125781159034,-1.231111883218897) circle (2.5pt);
\draw [fill=qqqqff] (6.370910095846811,0.5910375080896166) circle (2.5pt);
\draw [fill=qqqqff] (6.434877040051173,0.002541621409478362) circle (2.5pt);
\draw [fill=qqqqff] (8.639743589985672,-0.6535907548495615) circle (2.5pt);
\draw [fill=qqqqff] (8.744888255475496,0.3259148131345835) circle (2.5pt);
\draw[color=qqqqff] (8.866204964788267,0.6391050263898316) node {$C$};
\end{scriptsize}
\end{tikzpicture}
\end{adjustbox}
\end{minipage}
\end{frame}