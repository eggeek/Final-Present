\begin{frame}{Traditional k-Nearest Neighbor}
\setbeamercovered{invisible}
\begin{minipage}{.5\textwidth}
\small k-Nearest Neighbor:
\begin{itemize}
    \item<1-> \small Given:
    \begin{itemize}
        \item<2-> \small $q$: query point
        \item<3-> \small $T$: target set\\\tiny(e.g. $\{A,B,C,D\}$)
        \item<4-> \small $k$: number of retrieved targets \tiny(e.g. $k=1$)
    \end{itemize}
    \item<5-> \small Return:\\top $k$ nearest targets regarding \textit{Euclidean distance} $d_e$
    \item<6-> \small the circle indicates that $D$ is the nearest neighbor of $q$
\end{itemize}
\end{minipage}%
\begin{minipage}{.5\textwidth}
\begin{adjustbox}{max totalsize={.9\textwidth}{.9\textheight}, right}
\input{src/kNN.tex}
\end{adjustbox}
\end{minipage}
\end{frame}

\begin{frame}{Obstacle k-Nearest Neighbor}
\begin{minipage}{.5\textwidth}
\begin{itemize}
    \item \small traditional kNN has been well studied.
    \item \small when take obstacles into consideration...
    \item \small metric: Obstacle distance $d_o$
\end{itemize}
\end{minipage}%
\begin{minipage}{.5\textwidth}
\begin{adjustbox}{max totalsize={.9\textwidth}{.9\textheight}, right}
\only<1,2> {\input{src/kNN.tex}}
\only<3-> {\input{src/OkNN.tex}}
\end{adjustbox}
\end{minipage}
\end{frame}

\begin{frame}{Application Scenario}
\begin{minipage}{.5\textwidth}
\small In an industrial warehouse,\\
{\color{red}$q$} is a robot.\\
It's interested in the closest storage locations,\\
but it can not cross obstacles
\end{minipage}%
\begin{minipage}{.5\textwidth}
\begin{adjustbox}{max totalsize={.9\textwidth}{.9\textheight}, right}
\input{src/OkNN.tex}
\end{adjustbox}
\end{minipage}
\end{frame}